\documentclass{article}
\usepackage[margin=0.75in]{geometry}
\usepackage{amsmath}
\usepackage{graphicx}
\usepackage{pdfpages}

\graphicspath{{./LatexImg/}}

\begin{document}
	
	\includepdf{CoverPage.pdf}
	
	First and foremost, the programming language used to conduct this lab is Python 3.7, and the libraries primarily used are opencv2, numpy, and matplotlib. To view the cover page if it doesn't display properly, open \textbf{CoverPage.pdf} included in the folder provided.\\
	
	\section{Introduction}
	
	This lab examines the concept of artistic stylization and the different ways to produce post-processing images through certain filters and algorithms. The filters produced in this lab are effects that look unnatural. The two main concepts that will be discussed in this lab are the \textbf{eXtended Difference of Gaussians (XDOG)} and the \textbf{Cartoon Effect}.\\
	
	\section{Stylization Effects}
	
	Edges in digital photos are usually sharp transitions in intensities between the two sides of the edge. This phenomenon can be exploited to figure out where there could be salient edges within the photo based on the fluctuation in intensity between two points. One way to do this in a digital image is to approximate continuous differentiation to figure out the change in pixel intensity values, using Sobel operators. The Sobel operators are presented as kernels and are represented as such:\\
	
	\begin{align*}
	 h_{x}(x,y) = \left[
	 \begin{matrix}
	 	1 & 0 & -1\\
	 	2 & 0 & -2\\
	 	1 & 0 & -1
	 \end{matrix}
	 \right]
	\end{align*}
	
	\begin{align*}
		h_{y}(x,y) = \left[
		\begin{matrix}
			 1 & 2 & 1\\
			 0 & 0 & 0\\
			-1 & -2 & -1
		\end{matrix}
		\right]
	\end{align*}
	
	These masks must be convolved with the image to indicate the horizontal and vertical gradient of the image. Note that the second kernel is a transposed version of the first kernel. With these two transformation matrices, the next step in the procedure is to compute the gradient magnitude.
	
	\subsection{Gradient Magnitude}
	
	This step combines the two gradient matrices in a similar fashion to finding the gradient of a scalar field. The formula is listed below:\\
	
	\begin{equation}
		||\nabla I(x,y)|| = \sqrt{I_{x}(x,y))^2 + (I_{y}(x,y))^2}
	\end{equation}\\
	
	In this equation, the two image gradients, $I_{x}(x,y)$ and $ I_{y}(x,y)$ are the two matrices found in the previous process. These two matrices represent the derivatives in the horizontal and vertical directions. These are used in the equation to pinpoint where there might be a strong edge in a particular direction. The magnitude of these two gradients can be combined to find the edges in every direction, not just vertical and horizontal.
	
	\subsection{Non-Maximum Suppression}
	
	 A Non-Maximum Suppression filter is a non-linear filter that suppresses a value, by setting it to zero, if it is not the local maximum value within the surrounding values of its neighboring values.\\
	 
	 This is useful for edge detection as the intensities of edges often increase and decrease at various points and degrees. This is used after the change in intensities have been evaluated, therefore discerning the strong edges from the weak edges is used here.\\
	 
	 The equation for the NMS filter is shown below, with $w[n]$ being the spliced array and $w_{max}$ representing the max value in the array:\\
	 
	 \begin{equation}
	 NMS(w[n]) = \begin{cases} 
	 0, & \mbox{if}\ w[n] < w_{max}\\
	 w[n], & otherwise 
	 \end{cases} 
	 \end{equation}
	 \\
	
	This filter is ran twice, once in the horizontal direction and the other in the vertical direction. Once the two matrices are produced, the final step can be started.
	
	\subsection{Thresholding}
	
	The final step involves weeding out certain intensities, due to the values of those pixels being small and thus not referred to as a strong edge. The function takes an image, and threshold value and produces a binary image, $B(x,y)$, such that the image outlines the edges by setting them to the max value, and their surroundings to the min value.\\
	
	This is represent by the function below:\\
	
	\begin{equation}
		B(x,y) = \begin{cases} 
		1, & I(x,y) \geq T\\
		0, & I(x,y) < T
		\end{cases} 
	\end{equation}\\
	
	In the case of edge detection, the value of 1, or max value, indicates there is an edge at that index, and a 0 represents there is no such edge.\\
	
	Another step in the thresholding process is to merge the two binary images using an inclusive-or function. This will give the finished product combining both their edges to finalize the edge detection.
	
	\section{Assignment}
	
	\subsection{Filtering Functions}
	
	\subsubsection{Spatial Filter}
	
	The spatial filter is the core piece in the edge detection, and in any kernel filtering process. This convolves the kernel with the image array to produce a new transformed image, or array. The particular filter used in this lab uses a zero pad to avoid out of boundary indexing issues. However, this process does have an issue with the borders of the images as they are usually lower values due to the nature of the spatial filter and the zero padding contributing to the overall sum of the value.\\
	
	The Spatial Filter is shown in the code below:\\
	
	\noindent def spatial\textunderscore filter(F, W):\\
	\\
	\indent a,b = W.shape\\
	\indent m,n = F.shape\\
	\\
	\indent F\textunderscore double = np.array(F.copy(), np.double)\\
	\\
	\indent \# if filter kernel has size 1 by 1\\
	\indent if a == 1 \& b == 1:\\
	\\
	\indent \indent I = W*F\\
	\indent \indent return I\\
	\\
	\indent \# finding if column of kernel is odd or even and establishing the padding\\ 
	\indent \# parameters for the padded array\\
	\\
	\indent col\textunderscore right, col\textunderscore left = padding\textunderscore param(b)\\
	\indent row\textunderscore bottom, row\textunderscore top = padding\textunderscore param(a)\\
	\\
	\indent \# creating a padded array and an output for the convolution operation \\
	\indent F\textunderscore temp = np.zeros([m+col\textunderscore left+col\textunderscore right,n+row\textunderscore top+row\textunderscore bottom])\\
	\indent F\textunderscore temp[row\textunderscore top:m+row\textunderscore top, col\textunderscore left:n+col\textunderscore left] = 1.0\\
	\indent F\textunderscore temp[row\textunderscore top:m+row\textunderscore top, col\textunderscore left:n+col\textunderscore left] *= F\textunderscore double\\
	\indent I = np.zeros\textunderscore like(F\textunderscore double)\\
	\\
	\indent \# iterating over the length and width of the original size of the array\\
	\indent for i in range(row\textunderscore top,m+row\textunderscore top):\\
	\indent \indent for j in range(col\textunderscore left,n+col\textunderscore left):\\
	\\
	\indent \indent \indent sum = 0\\
	\\
	\indent \indent \indent \# partioning a section the same size as the kernel for the \\
	\indent \indent \indent \# convoltion operation and then computing the convolution and\\
	\indent \indent \indent \# storing it in the output array\\
	\\
	\indent \indent \indent snap = F\textunderscore temp[i-row\textunderscore top: i+row\textunderscore bottom+1, j-col\textunderscore left: j+col\textunderscore right+1].copy()\\
	\indent \indent \indent for l in range(0,len(W)):\\
	\indent \indent \indent \indent for k in range(0,len(W[l])):\\
	\indent \indent \indent \indent \indent sum += snap[l][k] * W[l][k]\\
	\indent \indent \indent I[i-row\textunderscore top][j-col\textunderscore left] = sum\\
	\\
	\indent return I\\
	
	It also uses the padding\textunderscore param function which determines the padding values for the image based on the size of the kernel. The code for the function is found below:\\
	
	\noindent def padding\textunderscore param(a):\\
	\\
	\indent if a \% 2 == 1: \\
	\indent \indent m = int((a - 1)/ 2)\\
	\indent \indent n = int(m)\\
	\\
	\indent else:\\
	\indent \indent	m = int(a / 2)\\
	\indent \indent n = int(m - 1)\\
	\\
	\indent return m,n\\
	
	The test image shown below in Image 1 was convolved with a Gaussian filter shown below in, to produce an blurry output image shown in Image 2:\\
	
	\begin{align*}
	h(x,y) = \frac{1}{331}\left[
	\begin{matrix}
	1 & 4 & 7 & 4 & 1\\
	4 & 20 & 33 & 20 & 4\\
	7 & 33 & 55 & 33 & 7\\
	4 & 20 & 33 & 20 & 4\\
	1 & 4 & 7 & 4 & 1
	\end{matrix}
	\right]
	\end{align*}\\
	
	\begin{center}
		\noindent \includegraphics[scale=0.75]{lena}\\
		Image 1: Input Image\\
		
		\includegraphics[scale=0.75]{s211o}\\
		Image 2: Output Image\\
	\end{center}
	
	\subsubsection{Non-Maximum Suppression}
	
	The Non-Maximum Filter functions the same way as described above in the introduction. The function takes in an image array, and a height and width of the NMS filtering window and outputs two filtered images, one with horizontal filtering and the other with vertical filtering.\\
	
	Furthermore, it uses the same zero padding algorithm as shown above to avoid any indexing issues. These zero values do not affect the overall output of the image as the pixel intensities are assumed to be all positive value and thus 0 is the minimum number possible. The code is shown below:\\
	
	\noindent def non\textunderscore max\textunderscore suppress(img, H, W):\\
	\\
	\indent m,n = img.shape\\
	\\
	\indent if H == 1 \& W == 1:\\
	\indent \indent return img\\
	\\
	\indent \# establishing the padding parameters for the padded array\\   
	\indent col\textunderscore right, col\textunderscore left = padding\textunderscore param(W)\\
	\indent row\textunderscore bottom, row\textunderscore top = padding\textunderscore param(H)\\
	\\
	\indent \# creating a padded array and an output for the max operation\\
	\indent F\textunderscore temp = np.zeros([m+col\textunderscore left+col\textunderscore right,n+row\textunderscore top+row\textunderscore bottom])\\
	\indent F\textunderscore temp[row\textunderscore top:m+row\textunderscore top, col\textunderscore left:n+col\textunderscore left] = 1.0\\
	\indent F\textunderscore temp[row\textunderscore top:m+row\textunderscore top, col\textunderscore left:n+col\textunderscore left] *= img\\
	\indent I\textunderscore vertical = np.zeros\textunderscore like(img)\\
	\indent I\textunderscore horizontal = np.zeros\textunderscore like(img)\\
	\\
	\indent \# iterating over the original array while applying a horizontal kernel\\
	\indent \# with a max value filter\\
	\indent for i in range(row\textunderscore top,m+row\textunderscore top):\\
	\indent \indent for j in range(col\textunderscore left,n+col\textunderscore left):\\
	\\
	\indent \indent \indent \# partitioning the array and computing the max value and storing\\
	\indent \indent \indent \# it in the output array\\
	\\
	\indent \indent \indent snap = F\textunderscore temp[i-row\textunderscore top: i+row\textunderscore bottom+1, j].copy()\\
	\indent \indent \indent if snap[row\textunderscore top] == np.amax(snap):\\
	\indent \indent \indent \indent I\textunderscore horizontal[i-row\textunderscore top,j-col\textunderscore left] = snap[row\textunderscore top]\\
	\\
	\indent \# same operation as above except with a vertical kernel\\
	\indent for i in range(row\textunderscore top,m+row\textunderscore top):\\
	\indent \indent for j in range(col\textunderscore left,n+col\textunderscore left):\\
	\\
	\indent \indent \indent snap = F\textunderscore temp[i, j-col\textunderscore left: j+col\textunderscore right+1].copy()\\
	\indent \indent \indent if snap[col\textunderscore left] == np.amax(snap):\\
	\indent \indent \indent \indent I\textunderscore vertical[i-row\textunderscore top,j-col\textunderscore left] = snap[col\textunderscore left]\\
	\\
	return I\textunderscore horizontal, I\textunderscore vertical\\
	
	This function was tested with Image 3 below to produce the following two images, one for the horizontal filtering and the other for vertical.\\
	
		\begin{center}
		\noindent \includegraphics[scale=0.75]{nms-test}\\
		Image 3: Input Image\\
		
		\includegraphics[scale=0.75]{s212ox}\\
		Image 4: Output Image, Horizontal\\
		
		\includegraphics[scale=0.75]{s212oy}\\
		Image 5: Output Image, Vertical\\
		
		\end{center}
	
	\subsubsection{Thresholding}
	
	The function for image thresholding is similar to the one mentioned above. This thresholding function takes the inputs of an image and a threshold ratio between the values of 0 and 1. The threshold value is then multiplied by the max value of the data type of the array and then is used to compare if each value above or below the threshold value and assigns the max value or 0 to the output binary image accordingly. The code for this function is displayed below:\\
	
	\noindent def image\textunderscore thresholding(img, T):\\
	\\
	\indent if 0 $>$ T or T $>$ 1:\\
	\indent \indent return img\\
	\\
	\indent max\textunderscore value = np.iinfo(img.dtype).max\\
	\\
	\indent \# Checks for the max value of the data type the image uses and multiplies\\
	\indent \# it with the ratio\\
	\indent T *= max\textunderscore value\\
	\\
	\indent B = np.zeros\textunderscore like(img)\\
	\\
	\indent for i in range(0,len(img)):\\
	\indent \indent for j in range(0,len(img[i])):\\
	\\
	\indent \indent \indent if img[i,j] $\leq$ T:\\
	\indent \indent \indent\indent 	B[i,j] = max\textunderscore value\\
	\indent return B\\
	
	The code was tested on the photo shown below to produce the image at varying threshold values, $T$.\\
	
	\begin{center}
		\noindent \includegraphics[scale=0.75]{threshold-test}\\
		Image 6: Input Image\\
		
		\includegraphics[scale=0.75]{s213o25}\\
		Image 7: Output Image, T = 25\% \\
		
		\includegraphics[scale=0.75]{s213o50}\\
		Image 8: Output Image, T = 50\% \\
		
		\includegraphics[scale=0.75]{s213o75}\\
		Image 9: Output Image, T = 75\% \\

	\end{center}
	
	\subsection{Edge Detector}
	
	The edge detector utilizes the functions described above to perform the pipeline outlined in the introduction to find the edges of any input images. The edge detector's inputs are an image array, a derivative kernel, an optional threshold value, and a window size for the NMS filter. \\
	
	The first stage of the edge detection is to remove the noise on the image whilst preserving the edges, which is done by using a gaussian filter. Next, the second step in the filter is to use the spatial filter with the kernel and a transposed matrix based of it to produce the horizontal and vertical gradients of the image. Afterwards, the magnitude of the image gradients is found by using equation 1, outlined above. Furthermore, the magnitude is put through the NMS filter to produce two images that outline the possible edges of the original image. Next, both those images are put into the threshold function to discern the strong edges from the weak ones. Finally, the combination of the two images through the inclusive or function unites the two images to produce the final product that displays the edges of the image file.\\
	
	The code for the edge detector is shown below:\\
	
	\noindent def edge\textunderscore detector(img, H, T=0.1, wndsz=5):\\
	\\
	\indent \# Converts the image to a grayscale if it isn't already\\
	\indent if(img.size == 3):\\
	\indent \indent I = cv2.cvtColor(img.copy(), cv2.COLOR\textunderscore BGR2GRAY)\\
	\indent else:\\
	\indent \indent I = img.copy()\\
	\\
	\indent \# Gaussian Kernel to reduce noise\\
	\indent g\textunderscore kernel = np.array([[1, 4, 7, 4, 1],\\
	\indent \indent [4, 20, 33, 20, 4],\\
	\indent \indent [7, 33, 55, 33, 7],\\
	\indent \indent [4, 20, 33, 20, 4],\\
	\indent \indent [1, 4, 7, 4, 1]])\\
	\\
	\indent g\textunderscore sum = np.sum(g\textunderscore kernel)  \\ 
	\\
	\indent \# Gets the transpose of the horizontal kernel to get the vertical kernel\\
	\indent H\textunderscore t = np.transpose(H)\\
	\\
	\indent \# used to reduce noise in the image while preserving edges\\
	\indent I = spatial\textunderscore filter(I, g\textunderscore kernel / g\textunderscore sum) \\ 
	\\
	\indent \# Convolutes the derivative approximation kernels to find the image \\
	\indent \# gradients\\
	\indent I\textunderscore x, I\textunderscore y = spatial\textunderscore filter(I, H), spatial\textunderscore filter(I, H\textunderscore t)\\
	\\
	\indent \# Computes the gradient magnitude\\
	\indent I = np.sqrt(I\textunderscore x**2 + I\textunderscore y**2)\\
	  \\
	\indent \# Suppressing low fluctuations in intensities\\
	\indent I\textunderscore x, I\textunderscore y = non\textunderscore max\textunderscore suppress(I, wndsz, wndsz)\\
	\\
	\indent \# Threshold all small values that indicate weak edges\\
	\indent \# I = image\textunderscore thresholding(np.array(I, np.uint8), T)\\
	\indent I\textunderscore x = image\textunderscore thresholding(np.array(I\textunderscore x, np.uint8), T)\\
	\indent I\textunderscore y = image\textunderscore thresholding(np.array(I\textunderscore y, np.uint8), T)\\
	\\
	\indent max\textunderscore value = np.iinfo(img.dtype).max\\
	\\
	\indent for i in range(0,len(I\textunderscore x)):\\
	\indent \indent for j in range(0, len(I\textunderscore x[i])):\\
	\\
	\indent \indent \indent if I\textunderscore x[i][j] == max\textunderscore value or I\textunderscore y[i][j] == max\textunderscore value:\\
	\indent \indent \indent \indent I[i][j] = max\textunderscore value\\
	\indent \indent \indent else:\\
	\indent \indent \indent I[i][j] = 0\\
	\\
	\indent return I\\
	
	The input image used is the same one in Image 1. The output image is shown below for both default parameters and manual parameters:\\
	
	\begin{center}
		\noindent \includegraphics[scale=0.75]{s22o-def}\\
		Image 10: Output Image, Default Parameters\\
		
		\includegraphics[scale=0.75]{s22o-adj}\\
		Image 11: Output Image, T = 30\%, window size = 7 \\
		
	\end{center}

	\subsection{Derivative Filters}
	
	The Derivative Filter function selects a derivative kernel for a user and outputs it for the user. The selector is given by the user as an input integer between 0 and 3. It has four premade kernels, including:\\
	
	\textbf{Central Difference}\\
	
	\begin{align*}
	h_{x}(x,y) = \left[
	\begin{matrix}
	1 & 0 & -1
	\end{matrix}
	\right]
	\end{align*}\\
	
	\textbf{Forward Difference}\\
	
	\begin{align*}
	h_{x}(x,y) = \left[
	\begin{matrix}
	0 & 1 & -1
	\end{matrix}
	\right]
	\end{align*}\\
	
	\textbf{Prewitt}\\
	
	\begin{align*}
	h_{x}(x,y) = \left[
	\begin{matrix}
	1 & 0 & -1\\
	2 & 0 & -2\\
	1 & 0 & -1
	\end{matrix}
	\right]
	\end{align*}
	
	\textbf{Sobel}\\
	
	\begin{align*}
	h_{x}(x,y) = \left[
	\begin{matrix}
	1 & 0 & -1\\
	2 & 0 & -2\\
	1 & 0 & -1
	\end{matrix}
	\right]
	\end{align*}
	
	The code for the derivative filter function is displayed below:\\
	
	\noindent def derivative\textunderscore kernel(select):\\
	\\
	\indent \# checks for boundary conditions\\
	\indent select = int(select)\\
	\indent if  select $>$ 3 or select $<$ 0:\\
	\indent \indent return 0\\
	\\
	\indent \# selects operation\\
	\indent if select == 0:\\
	\indent \indent h = np.array([1, 0, -1])\\
	\\
	\indent elif select == 1:\\
	\indent \indent h = np.array([0, 1, -1])\\
	\\
	\indent elif select == 2:\\
	\indent \indent h = np.array([[1, 0, -1],\\
	\indent \indent [1, 0, -1],\\
	\indent \indent [1, 0, -1]])\\
	\\
	\indent else:\\
	\indent \indent h = np.array([[1, 0, -1],\\
	\indent \indent [2, 0, -2],\\
	\indent \indent [1, 0, -1]])\\
	\\
	\indent return h\\
	
	The images below represent the different kernels that were used when sent to the edge detector function above with the same input image, except the Sobel kernel as that was done before in the first test.\\
	
		\begin{center}
		\noindent \includegraphics[scale=0.75]{s23o2}\\
		Image 12: Output Image, Prewitt\\
		
	\end{center}
	
	\section{Analysis}
	
	\begin{enumerate}
		\item When a filter is completely separable, the kernel matrix is a called a rank 1. The rank of a matrix is the number of linear independent column or row vectors.\\ 
		
		Since the kernel is separable it means that the convolution operation can be performed by convoluting the vector kernels in a horizontal and vertical manner to emmulate the $N \times N$ matrix. This helps speed up the algorithm by doing less calculations as instead of doing, $N \times N$ operations per pixel, the computer is doing $N + N$ operations due to the amount of pixels being indexed.
		
		\item Applying the NMS filter with two seperate one-dimensional windows enables the horizontal and vertical edges to be better detected by the algorithm as the lines are compared with the changes in intensity in one direction only, as opposed to all directions.\\
		
		When computing the NMS with a two-dimensional filter, the image gradients are compared against the directional changes in all directions. Therefore the edges would be outlined, but they would not look as defined as they appear to be more segmented compared to the previous method described.\\
		
		\begin{center}
			\noindent \includegraphics[scale=0.75]{s22o-2D}\\
			Image 12: Edge Detection, 2D NMS Filter\\
		\end{center}
		
		\item A pre-processing step before implementing the edge detection algorithm is by reducing the noise. As stated above in the implementation used in this lab, the image was subjected to a Gaussian filter to reduce the overall noise in the image, because small unwanted spikes in intensities could cause the filter to accidentally detect that discrepancy as part of an edge.\\
		
		The central and forward difference did not work in the spatial filter function, therefore there cannot be a comparison on the images. 
		
	\end{enumerate}
	
\end{document}