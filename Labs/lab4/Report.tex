\documentclass{article}
\usepackage[margin=0.75in]{geometry}
\usepackage{amsmath}
\usepackage{graphicx}
%\usepackage{pdfpages}

\graphicspath{{./LatexImg/}}

\begin{document}
	
	%\includepdf{CoverPage.pdf}
	
	First and foremost, the programming language used to conduct this lab is Python 3.7, and the libraries primarily used are opencv2, numpy, and matplotlib. To view the cover page if it doesn't display properly, open \textbf{CoverPage.pdf} included in the folder provided.\\
	
	\section{Introduction}
	
	This lab examines the concept of artistic stylization and the different ways to produce post-processing images through certain filters and algorithms. The filters produced in this lab are effects that look unnatural. The two main concepts that will be discussed in this lab are the \textbf{eXtended Difference of Gaussians (XDOG)} and the \textbf{Cartoon Effect}.\\
	
	\section{Stylization Effects}
	
	\subsection{eXtended Difference of Gaussian}
	
	The eXtended Difference of Gaussian filter is a simple tool that uses second-order derivative filtering and thresholding.\\
	
	The first step in XDOG, is to find difference between two Gaussian blurred images, one with a standard deviation of $\sigma$, and the other is a scaled version, $k\sigma$, and the sum between that difference with the unscaled Gaussian blurred image. The equation below describes the XDOG filter:\\
	
	\begin{equation}
	D_{\sigma,k,p}(x,y) = (1 + p)G_{\sigma}(x,y) - pG_{k\sigma}(x,y)
	\end{equation}\\
	
	The $k$ term describes the second image's blurring kernel coefficient that controls the level of blur. The only restriction of k is that it is a positive real number, k $>$ 0. The $p$ term describes the potency of the Difference of Gaussian; a higher value means that the difference part affects the original blur more. The values of p have no set range, however if the value exceeds a certain point, the values may experience overflow, thus ruining the intended output. The equation can be rearranged by factoring the values influenced by $p$ in equation 1 to show a more clear definition of the XDOG filter:\\
	
	\begin{equation}
	D_{\sigma,k,p}(x,y) = G_{\sigma}(x,y) + p(G_{\sigma}(x,y) - G_{k\sigma}(x,y) ) 
	\end{equation}\\
	
	By rearranging the equation, it is clear to see that the effect of this filter is a unsharp mask. The only main difference between this process and a normal unsharp mask is that the image is blurred initially, and then the edge image, the difference part of Gaussian, is added back to provide the edge details of the image.\\
	
	The next and final step is to threshold the image to produce the output image. Two methods were implemented in this lab, they can be referred to as hard and soft thresholding.\\
	
	Hard thresholding uses a threshold value, and sets any intensity value less than or equal to the value to 0 and any value greater than to 1. The equation below displays the definition:\\
	
	\begin{equation}
	T(u) = \begin{cases} 
	1, & u > \epsilon\\
	0, & u \leq \epsilon
	\end{cases} 
	\end{equation}\\
	
	Soft Thresholding behaves slightly differently to the hard threshold, where there can be only two values for hard thresholding, there is a transistion between 0 and 1 for the soft thresholding method. This method uses the hyperbolic tangent function to determine the output value of the image if it less than the threshold value.\\
	
	 \begin{equation}
	 T_{\phi}(u) = \begin{cases} 
	 1, & u > \epsilon\\
	 1 + tanh(\phi(u - \epsilon)), & u \leq \epsilon
	 \end{cases} 
	 \end{equation}\\
	
	The value, $\epsilon $ controls the sharpness of the transition between the 0 and 1 value.\\ 
	
	\subsection{Cartoon Effect}
	
	A non-organic filter that adds an effect to a digital image that gives the appearance of an image that has been painted by smearing and giving an effect of mixing colours. Additional effects can be added by adding the edges, the difference of Gaussians part, onto the effect described earlier, to finalize the stylization.
	
	\section{Assignment}
	
	\subsection{XDOG}
	
	\subsubsection{XDOG Kernel}
	
	The eXtended Difference of Gaussians filter implementation follows the same guidelines as equation 1 and 2 and the descriptions above. The following filter is strictly a grayscale filter and any coloured images are converted at the beginning the code below.\\
		
	\noindent \# eXtended Difference of Gaussian\\
	\noindent def XDOG(img, k, sigma, p):\\
	\\
	\indent img\textunderscore temp = img.copy()\\
	\\
	\indent \#    Check for RGB image, if so then convert to grayscale\\
	\indent if img\textunderscore temp.ndim == 3:\\
	\indent \indent img\textunderscore temp = cv2.cvtColor(img\textunderscore temp, cv2.COLOR\textunderscore BGR2GRAY)\\
	\\
	\indent img\textunderscore temp = np.array(img\textunderscore temp, np.double)\\
	\\
	\indent \#    Creation of the gaussian kernels\\
	\indent gauss\textunderscore 1 = gaussian\textunderscore kernel(sigma, 2)\\		\indent gauss\textunderscore 2 = gaussian\textunderscore kernel(k * sigma, 2)\\
	\\
	\indent gauss\textunderscore 1\textunderscore sum = np.sum(gauss\textunderscore 1)\\
	\indent gauss\textunderscore 2\textunderscore sum = np.sum(gauss\textunderscore 2)\\
	\\
	\indent \#    Both gaussian kernels are applied to the image\\
	\indent G1 = bf.spatial\textunderscore filter(img\textunderscore temp, \indent gauss\textunderscore 1 / gauss\textunderscore 1\textunderscore sum)\\
	\indent G2 = bf.spatial\textunderscore filter(img\textunderscore temp, \indent gauss\textunderscore 2 / gauss\textunderscore 2\textunderscore sum)\\
	\\
	\indent \#    Difference of Gaussian Computation\\
	\indent img\textunderscore temp[:,:] = (1 + p) * G1[:,:] - p * G2[:,:]\\
	\\
	\indent return np.array(img\textunderscore temp, np.uint8)\\
		
	These are the input and output images for various values of k, p, and $\sigma$:\\
		
	%\begin{center}
	%	\noindent \includegraphics[scale=0.75]{lena}\\
	%	Image 1: Input Image\\
			
	%	\includegraphics[scale=0.75]{s211o}\\
	%	Image 2: Output Image\\
	%\end{center}

	\subsubsection{Threshold}
		
	The hard and soft threshold functions found in equations 3 and 4 are implemented below. The images below differ from the equations described above as they are not binary images, but images that range across the unsigned 8-bit spectrum.\\
		
	\smallskip	
	
	\noindent \textbf{Soft Thresholding}\\
		
	\noindent \# Produces binary threshold image\\
	\noindent def hard\textunderscore threshold(img, cutoff):\\
	\\
	\indent out = np.zeros\textunderscore like(img)\\
	\\
	\indent \# applies the max value to the intensities greater than the cutoff\\
	\indent out[img $>$ cutoff] = 255\\
	\\	
	\indent return out\\
	
	Output images for the images produced in the XDOG filter above.\\
	
%	\begin{center}
%		\noindent \includegraphics[scale=0.75]{nms-test}\\
%		Image 3: Input Image\\
%		
%		\includegraphics[scale=0.75]{s212ox}\\
%		Image 4: Output Image, Horizontal\\
%		
%		\includegraphics[scale=0.75]{s212oy}\\
%		Image 5: Output Image, Vertical\\
%	\end{center}
	
	\smallskip
	
	\textbf{Soft Threshold}\\
	
	\noindent \# Implements a soft thresholding function\\
	def soft\textunderscore thereshold(img, cutoff, phi):\\
	\\
	\indent out = np.array(np.ones\textunderscore like(img), np.double)\\
	\\
	\indent \# apply the threshold function\\
	\indent out[img <= cutoff] = 1 + np.tanh(phi * (img[i][j]) - cutoff)\\
	\\
	\indent return np.array( out * 255, np.uint8)\\
	
	Output images for the images produced in the XDOG filter above.\\
	
%	\begin{center}
%		\noindent \includegraphics[scale=0.75]{nms-test}\\
%		Image 3: Input Image\\
%		
%		\includegraphics[scale=0.75]{s212ox}\\
%		Image 4: Output Image, Horizontal\\
%		
%		\includegraphics[scale=0.75]{s212oy}\\
%		Image 5: Output Image, Vertical\\
%	\end{center}
	
	\subsection{Three-Tone Operator}
	
	For the three tone generator, the same method that was in the soft thresholding was applied to this operator. Instead of the hyperbolic tangent function being applied when the threshold wasn't exceeded, the function is applied globally. The code for the function is shown below:\\
	
	\noindent \# Three tone generator\\
	\noindent def three\textunderscore tone(img, cutoff, phi):\\
	\\
	\indent out = np.array(np.ones\textunderscore like (img), np.double)\\
	\\
	\indent for i in range(0, len(img)):\\
	\indent \indent for j in range(0, len(img[i])):\\
	\\
	\indent \indent \indent out[i][j] = 1 + np.tanh(phi * (img[i][j] - cutoff) )\\
	\\
	\indent return np.array( (out * 127) + 1, np.uint8)\\
	
	Output images for the images produced in the XDOG filter above.\\
	
%	\begin{center}
%		\noindent \includegraphics[scale=0.75]{nms-test}\\
%		Image 3: Input Image\\
%		
%		\includegraphics[scale=0.75]{s212ox}\\
%		Image 4: Output Image, Horizontal\\
%		
%		\includegraphics[scale=0.75]{s212oy}\\
%		Image 5: Output Image, Vertical\\
%	\end{center}
	
	\subsection{Oilify Filter}
	
	Need to add.
	
	\subsection{Cartoon Effect}
	
	\subsubsection{Edge Preserving, Smoothing Area Filter}
	
	This filter was implemented to smooth out the image, to produce a cartoon blur effect while preserving the edges. This filter was achieved by using a median filter and expanding the dimensions of the kernel every iteration. This function allows the edges to be preserved as the median filter will not filter out strong edges due to the complete difference in intensities between both sides of the edge. Consequently, it allows the overall image to be smoothed out as a whole, achieving the desired effect.\\
	
	\noindent def edge\textunderscore preserving(img, min\textunderscore window\textunderscore size, iteration):\\
	\\
	\indent out\textunderscore img = img.copy()\\
	\\
	\indent \#     Check for RGB image, if so then convert to grayscale\\
	\indent if out\textunderscore img.ndim == 3:   \\      
	\indent \indent out\textunderscore img = cv2.cvtColor(out\textunderscore img, cv2.COLOR\textunderscore BGR2GRAY)\\
	\\
	\indent for i in range(0, iteration):\\
	\indent \indent out\textunderscore img = cv2.medianBlur(out\textunderscore img, min\textunderscore window\textunderscore size + 2 * i)\\
	\\
	\indent return out\textunderscore img\\
	
	The filter produced the following output with the input provided below:\\
	
%	\begin{center}
%		\noindent \includegraphics[scale=0.75]{lena}\\
%		Image 1: Input Image\\
%		
%		\includegraphics[scale=0.75]{s211o}\\
%		Image 2: Output Image\\
%	\end{center}
	
	\subsubsection{Edge Detector}
	
	The edge detector's purpose is to discover the edges in a pleasant manner in order to implement it in the cartoon effect. \\
	
	The first stage of the edge detection is to remove the noise on the image whilst preserving the edges, which is done by using a Gaussian filter. Next, the second step in the filter is to use the spatial filter with the kernel and a transposed matrix based of it to produce the horizontal and vertical gradients of the image. The filter used in this investigation is the Sobel filter. Afterwards, the magnitude of the image gradients is found by using equation 1, outlined above. Furthermore, the magnitude is put through the NMS filter to produce two images that outline the possible edges of the original image. Next, both those images are put into the threshold function to discern the strong edges from the weak ones by assigning two different intensity values if it is a weak or strong edge. Next, the combination of the two images through an iterative loop looks for similarities between the two images by equating both images pixel by pixel to compare if both values are the same and if they are then the output of the function is the value of both of the image pixel intensities. The final step, Hysteresis, involves connecting the weak edges to the strong edges to show a more consistent tone. This is done by iterating through the image 4 times every direction. The algorithm will use the weak edge pixel and look at all 8 neighbouring pixels for a strong edge and replace the existing pixel with a strong pixel intensity (max value) if it finds a surrounding pixel. Each of the four direction matrices are then summed into a new matrix. The matrix is then put into a comparator to see if any of the values of the four matrices are strong edges, and sets the final image to a strong edge at that particular pixel.\\
	
	\noindent The code for the edge detector is shown below:\\
	
	\noindent \textbf{Edge Extract Main}
	
	\smallskip
	
	\noindent def extract\textunderscore edges(img, sigma = 0.33):\\
	\\
	\indent out\textunderscore img = np.zeros\textunderscore like(img)\\
	\\
	\indent \# Parameter for auto thresholding\\
	\indent med = np.median(img)\\
	\\
	\indent \# High and Low Auto Threshold Values\\
	\indent lo\textunderscore thresh = int(max(0, (1.0 - sigma) * med))\\
	\indent hi\textunderscore thresh = int(min(255, (1.0 + sigma) * med))\\
	\\
	\indent \# Sobel Filter\\
	\indent H = np.array([[1, 0, -1],\\
	\indent \indent [2, 0, -2],\\
	\indent \indent [1, 0, -1]])\\
	\\
	\indent \# Transposed Sobel\\
	\indent H\textunderscore t = np.transpose(H)\\
	\\
	\indent \# Gaussian Kernel\\
	\indent gauss = gaussian\textunderscore kernel(1, 2)\\
	\indent gauss\textunderscore sum = np.sum(gauss)\\
	\\
	\indent \# Noise Reduction Step, Gaussian Blur\\
	\indent img\textunderscore temp = bf.spatial\textunderscore filter(img, gauss / gauss\textunderscore sum)\\
	\\
	\indent I\textunderscore x = bf.spatial\textunderscore filter(img\textunderscore temp, H)\\
	\indent I\textunderscore y = bf.spatial\textunderscore filter(img\textunderscore temp, H\textunderscore t)\\
	\\
	\indent \# Compute gradient magnitude\\
	\indent I = np.sqrt(I\textunderscore x**2 + I\textunderscore y**2)\\
	\\
	\indent \# Suppressing low fluctuations in intensities\\
	\indent I\textunderscore x, I\textunderscore y = bf.non\textunderscore max\textunderscore suppress(I, 5, 5)\\
	\\
	\indent I\textunderscore x = threshold(I\textunderscore x, hi\textunderscore thresh, lo\textunderscore thresh)\\
	\indent I\textunderscore y = threshold(I\textunderscore y, hi\textunderscore thresh, lo\textunderscore thresh)\\
	\\
	\indent for i in range(0,len(I\textunderscore x)):\\
	\indent \indent for j in range(0, len(I\textunderscore x[i])):\\
	\\
	\indent \indent \indent if I\textunderscore x[i,j] == 255 or I\textunderscore y[i,j] == 255:\\
	\indent \indent \indent \indent out\textunderscore img[i,j] = 255\\
	\\
	\indent \indent \indent elif I\textunderscore x[i,j] == 50 or I\textunderscore y[i,j] == 50:\\
	\indent \indent \indent \indent out\textunderscore img[i,j] = 50\\
	\\
	\indent \indent \indent elif(I\textunderscore x[i,j] == 50 or I\textunderscore y[i,j] == 255) or (I\textunderscore x[i,j] == 255 or I\textunderscore y[i,j] == 50):\\
	\indent \indent \indent \indent out\textunderscore img[i,j] = 50\\
	\\
	\indent out\textunderscore img = hysteresis(out\textunderscore img)\\
	\\
	\indent return np.array(out\textunderscore img, np.uint8)\\
	
	\smallskip
	
	\noindent \textbf{Thresholding}
	
	\smallskip
	
	\noindent def threshold(img, hi, lo):\\
	\\
	\indent \# Values for strong and weak edges\\
	\indent high = 255\\
	\indent low = 50\\
	\\
	\indent out\textunderscore img = np.zeros\textunderscore like(img)\\
	\\
	\indent strong\textunderscore x, strong\textunderscore y = np.where(img $geq$ hi)\\
	\indent weak\textunderscore x, weak\textunderscore y = np.where((img $\leq$ hi) and (img $\geq$ lo))\\
	\\
	\indent out\textunderscore img[strong\textunderscore x, strong\textunderscore y] = high\\
	\indent out\textunderscore img[weak\textunderscore x, weak\textunderscore y] = low\\
	\\
	\indent return np.array(out\textunderscore img, np.uint8)\\
	
	\smallskip
	
	\textbf{Hysteresis}
	
	This function below was influenced greatly by Abhisek Jana's implementation[1] of this function.\\
	
	\noindent def hysteresis(image):\\
	\\
	\indent weak = 50\\
	\\
	\indent image\textunderscore row, image\textunderscore col = image.shape\\
	\\
	\indent top\textunderscore to\textunderscore bottom = image.copy()\\
	\\
	\indent for row in range(1, image\textunderscore row - 1):\\
	\indent \indent for col in range(1, image\textunderscore col - 1):\\
	\indent \indent \indent if top\textunderscore to\textunderscore bottom[row, col] == weak:\\
	\\
	\indent \indent \indent \indent if top\textunderscore to\textunderscore bottom[row, col + 1] == 255 or top\textunderscore to\textunderscore bottom[row, col - 1] == 255 or \\
	\indent \indent \indent \indent   top\textunderscore to\textunderscore bottom[row - 1, col] == 255 or top\textunderscore to\textunderscore bottom[row + 1, col] == 255 or \\
	\indent \indent \indent \indent   top\textunderscore to\textunderscore bottom[row - 1, col - 1] == 255 or top\textunderscore to\textunderscore bottom[row + 1, col - 1] == 255 or\\
	\indent \indent \indent \indent   top\textunderscore to\textunderscore bottom[row - 1, col + 1] == 255 or top\textunderscore to\textunderscore bottom[row + 1, col + 1] == 255: \\
	\\
	\indent \indent \indent \indent \indent top\textunderscore to\textunderscore bottom[row, col] = 255\\
	\\
	\indent \indent \indent \indent else:\\
	\indent \indent \indent \indent \indent     top\textunderscore to\textunderscore bottom[row, col] = 0\\
	\\
	\indent bottom\textunderscore to\textunderscore top = image.copy()\\
	\\
	\indent for row in range(image\textunderscore row - 2, 1, -1):\\
	\indent \indent for col in range(image\textunderscore col - 2, 1, -1):\\
	\indent \indent \indent if bottom\textunderscore to\textunderscore top[row, col] == weak:\\
	\\
	\indent \indent \indent \indent if bottom\textunderscore to\textunderscore top[row, col + 1] == 255 or bottom\textunderscore to\textunderscore top[row, col - 1] == 255 or \\
	\indent \indent \indent \indent   bottom\textunderscore to\textunderscore top[row - 1, col] == 255 or bottom\textunderscore to\textunderscore top[row + 1, col] == 255 or\\
	\indent \indent \indent \indent   bottom\textunderscore to\textunderscore top[row - 1, col - 1] == 255 or bottom\textunderscore to\textunderscore top[row + 1, col - 1] == 255 or\\
	\indent \indent \indent \indent   bottom\textunderscore to\textunderscore top[row - 1, col + 1] == 255 or bottom\textunderscore to\textunderscore top[row + 1, col + 1] == 255:\\
	\\
	\indent \indent \indent \indent \indent bottom\textunderscore to\textunderscore top[row, col] = 255\\
	\\
	\indent \indent \indent \indent else:\\
	\indent \indent \indent \indent \indent bottom\textunderscore to\textunderscore top[row, col] = 0\\
	\\
	\indent right\textunderscore to\textunderscore left = image.copy()\\
	\\
	\indent for row in range(1, image\textunderscore row - 1):\\
	\indent \indent for col in range(image\textunderscore col - 2, 0, -1):\\
	\indent \indent \indent if right\textunderscore to\textunderscore left[row, col] == weak:\\
	\\
	\indent \indent \indent \indent if right\textunderscore to\textunderscore left[row, col + 1] == 255 or right\textunderscore to\textunderscore left[row, col - 1] == 255 or \\
	\indent \indent \indent \indent   right\textunderscore to\textunderscore left[row - 1, col] == 255 or right\textunderscore to\textunderscore left[row + 1, col] == 255 or \\
	\indent \indent \indent \indent   right\textunderscore to\textunderscore left[row - 1, col - 1] == 255 or right\textunderscore to\textunderscore left[row + 1, col - 1] == 255 or \\
	\indent \indent \indent \indent   right\textunderscore to\textunderscore left[row - 1, col + 1] == 255 or right\textunderscore to\textunderscore left[row + 1, col + 1] == 255:\\
	\\
	\indent \indent \indent \indent \indent right\textunderscore to\textunderscore left[row, col] = 255\\
	\indent \indent \indent \indent else:\\
	\indent \indent \indent \indent \indent right\textunderscore to\textunderscore left[row, col] = 0\\
	\\
	\indent left\textunderscore to\textunderscore right = image.copy()\\
	\\
	\indent for row in range(image\textunderscore row - 2, 1, -1):\\
	\indent \indent for col in range(1, image\textunderscore col - 2):\\
	\indent \indent \indent if left\textunderscore to\textunderscore right[row, col] == weak:\\
	\\
	\indent \indent \indent \indent if left\textunderscore to\textunderscore right[row, col + 1] == 255 or left\textunderscore to\textunderscore right[row, col - 1] == 255 or \\
	\indent \indent \indent \indent   left\textunderscore to\textunderscore right[row - 1, col] == 255 or left\textunderscore to\textunderscore right[row + 1, col] == 255 or \\
	\indent \indent \indent \indent   left\textunderscore to\textunderscore right[row - 1, col - 1] == 255 or left\textunderscore to\textunderscore right[row + 1, col - 1] == 255 or \\
	\indent \indent \indent \indent   left\textunderscore to\textunderscore right[row - 1, col + 1] == 255 or left\textunderscore to\textunderscore right[row + 1, col + 1] == 255:\\
	\indent \indent \indent \indent left\textunderscore to\textunderscore right[row, col] = 255\\
	\indent \indent \indent \indent else:\\
	\indent \indent \indent \indent \indent left\textunderscore to\textunderscore right[row, col] = 0\\
	\\
	\indent out\textunderscore img = top\textunderscore to\textunderscore bottom + bottom\textunderscore to\textunderscore top + right\textunderscore to\textunderscore left + left\textunderscore to\textunderscore right\\
	\\
	\indent for i in range(0,len(out\textunderscore img)):\\
	\indent \indent for j in range(0,len(out\textunderscore img[i])):\\
	\\
	\indent \indent \indent if(out\textunderscore img[i,j]):\\
	\indent \indent \indent \indent out\textunderscore img[i,j] = 255\\
	\\
	\indent return out\textunderscore img\\
	
	The Spatial Filtering, NMS Function and Gradient operations remain unchanged from the functions described in Lab 2.\\
	
	The Output for this functions, for this specific input is:\\
	
%	\begin{center}
%		\noindent \includegraphics[scale=0.75]{s22o-def}\\
%		Image 10: Output Image, Default Parameters\\
%		
%		\includegraphics[scale=0.75]{s22o-adj}\\
%		Image 11: Output Image, T = 30\%, window size = 7 \\
%		
%	\end{center}

	\subsection{Cartoon Effect}

	The cartoon effect utilizes both functions described above to produce the effect. This is done by computing both the filters and setting the output image to the smoothed image. Finally, the image edge pixels, found by the edge image, are replaced by an averaging filter, only taking the neighbouring 8 pixels into the equation, and not the centre pixel.\\
	
	The code below describes the following effect:\\
	
	\noindent def cartoon\textunderscore effect(img, min\textunderscore window\textunderscore size, iteration, sigma = 0.33):\\
	\\
	\indent \# Generate the edge detect image\\
	\indent img\textunderscore edge = extract\textunderscore edges(img, sigma = sigma)\\
	\\
	\indent \# Generate the smoothing filter\\
	\indent img\textunderscore blur = edge\textunderscore preserving(img, min\textunderscore window\textunderscore size, iteration)\\
	\\
	\indent out\textunderscore img = img\textunderscore blur\\
	\\
	\indent for i in range(1, len(img) - 1):\\
	\indent \indent for j in range(1, len(img[i]) - 2):\\
	\\
	\indent \indent \indent if(img\textunderscore edge[i, j] == 255):\\
	\\
	\indent \indent \indent \indent \# Calculate the average surrounding pixel intensities except the edge pixel\\
	\indent \indent \indent \indent sum = np.sum(img[i-1:i+2, j-1:j+2])\\
	\indent \indent \indent \indent out\textunderscore img[i,j] = (sum - img[i,j]) / 8\\
	\\
	\indent return out\textunderscore img\\

	The output images are as follows for the input image and parameters:\\

	\subsection{Cartoon Effect, XDOG Modificaiton}
	
	This effect is similar to the Cartoon Effect described above, however, the XDOG effect is applied to every pixel on the image if the mode is selected. This amplifies the cartoon effect and also adds more details to the photo as well after the smoothing function.\\
	
	The code for the above explanation is as follows:\\
	
	\noindent def cartoon\textunderscore effect(img, min\textunderscore window\textunderscore size, iteration, sigma = 0.33, k = 1.4, p = 1, sigma\textunderscore x = 1, flag = 0):\\
	\\
	\indent img\textunderscore edge = extract\textunderscore edges(img, sigma = sigma)\\
	\\
	\indent img\textunderscore blur = edge\textunderscore preserving(img, min\textunderscore window\textunderscore size, iteration)\\
	\\
	\indent out\textunderscore img = img\textunderscore blur\\
	\\
	\indent \# Incorporation of the XDOG function\\
	\indent if(flag == 1):\\
	\indent \indent \#       Creation of the gaussian kernels\\
	\indent \indent gauss\textunderscore 1 = gaussian\textunderscore kernel(sigma\textunderscore x, 2)\\
	\indent \indent gauss\textunderscore 2 = gaussian\textunderscore kernel(k * sigma\textunderscore x, 2)\\
	\\
	\indent \indent gauss\textunderscore 1\textunderscore sum = np.sum(gauss\textunderscore 1)\\
	\indent \indent gauss\textunderscore 2\textunderscore sum = np.sum(gauss\textunderscore 2)\\
	\\
	\indent \indent \# Both gaussian kernels are applied to the image\\
	\indent \indent G1 = bf.spatial\textunderscore filter(img, gauss\textunderscore 1 / gauss\textunderscore 1\textunderscore sum)\\
	\indent \indent G2 = bf.spatial\textunderscore filter(img, gauss\textunderscore 2 / gauss\textunderscore 2\textunderscore sum)\\
	\\
	\indent for i in range(1, len(img) - 1):\\
	\indent \indent for j in range(1, len(img[i]) - 2):\\
	\\
	\indent \indent \indent if(img\textunderscore edge[i, j] == 255):\\
	\\
	\indent \indent \indent \indent \# Calculate the average surrounding pixel intensities except the edge pixel\\
	\indent \indent \indent \indent sum = np.sum(img[i-1:i+2, j-1:j+2])\\
	\indent \indent \indent \indent out\textunderscore img[i,j] = (sum - img[i,j]) / 8\\
	\\
	\indent \indent \indent \# Implementing the XDOG function if the flag has been set\\
	\indent \indent \indent if(flag == 1):\\
	\indent \indent \indent \indent out\textunderscore img[i,j] += p * (G1[i,j] - G2[i,j])\\
	\\
	\indent return out\textunderscore img\\
	
	The output for the following function using the input image and the following parameters is as follows:\\
	
	
	
	\clearpage
	
	\section{References}
	
	[1] A. Jana, “Implement Canny edge detector using Python from scratch,” A Developer Diary, 20-May-2019. [Online]. Available: http://www.adeveloperdiary.com/data-science/computer-vision/implement-canny-edge-detector-using-python-from-scratch/. [Accessed: 16-Nov-2019].
	
\end{document}