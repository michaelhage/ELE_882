\documentclass{article}
\usepackage[margin=1in]{geometry}
\usepackage{amsmath}

\begin{document}

\section{Assignment}

\subsection{Point Operation}

Point operations deal with the individual manipulation of pixel intensities within the image. An example of a point operation would be \newline


$I_{p}(x,y) = CI(x,y) + B$ [1] \newline


This particular transformation would scale the intensities of each pixel ($I(x,y)$) within a range, by a factor of $C$ and then offsets the value by a value $B$. \newline

It is important to note that the intensity value of an image cannot exceed the boundaries of 0 or 255. If an overflow occurs then the resultant value that will be stored is the remainder of the result divided by 256.

\subsubsection{Problem 1}

This problem is the implementation of the formula portrayed in equation 1. The value of $C$ can be any real value, and the value of $B$ must be within the range of 0-255. The following equation was implemented using the following code: \newline

\noindent def apply\textunderscore point\textunderscore tfrm(in\textunderscore img, c, b): \newline
	\indent \# this copies the image independent of the original image \newline
\indent	out\textunderscore img = copy.copy(in\textunderscore img)\newline
\newline
\indent	\# iterates through the image \newline
\indent	for i in range(len(in\textunderscore img)):\newline
\indent \indent	for j in range(len(in\textunderscore img[i])):\newline
\newline
\indent \indent \indent		\# applies the transform to a value \newline
\indent \indent \indent		x = (c * in\textunderscore img[i][j]) + b \newline
\newline
\indent \indent \indent		\# checks for overflow \newline
\indent \indent \indent		if x > 255: \newline
\indent \indent \indent \indent	x = 255 \newline
\indent \indent \indent 	elif x < 0: \newline
\indent \indent \indent \indent	x = 0 \newline

\indent \indent \indent		\# applies the value to the image pixel \newline
\indent \indent \indent		out\textunderscore img[i][j] = x \newline
\newline
\indent	return out\textunderscore img \newline

When the following transform is applied to an image they produce the following results: \newline

insert image

\subsubsection{Problem 2}

This transformation is a type of masking transformation that combines the two images, $ I_{A}(x,y), I_{B}(x,y)$, by comparing the values of a third mask image, $ M(x,y) $. This particular one uses a mask image to decide which image pixel to append to the output image. It does so through the following equation: \newline

\[ I_{p}(x,y) = \begin{cases} 
I_{A}(x,y) & M(x,y) > 0\\
I_{B}(x,y) & M(x,y) = 0  \indent [2]
\end{cases}
\] \newline

\noindent The function is implemented using the following code: \newline

\noindent def apply\textunderscore mask(img\textunderscore a, img\textunderscore b, img\textunderscore mask): \newline
\newline
\indent \# copies the first image \newline
\indent \indent out\textunderscore img = copy.copy(img\textunderscore a) \newline

\indent \# iterates through the image \newline
\indent for i in range(len(img\textunderscore a)): \newline
\indent \indent for j in range(len(img\textunderscore a[i])): \newline
\newline
\indent \indent \indent\# if mask image pixel is black, then replace pixel with the second image \newline
\indent \indent \indent if img\textunderscore mask[i][j] == 0: \newline
\newline
\indent \indent \indent \indent out\textunderscore img[i][j] = img\textunderscore b[i][j] \newline
\newline
\indent return out\textunderscore img \newline



\end{document} 